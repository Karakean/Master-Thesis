\chapter{Educational demonstrator}
\label{chap:EducationalDemonstrator}
This chapter presents an educational demonstrator designed to help students understand how anonymous communication networks work in practice and to illustrate the effects of the analysis performed in this work. The demonstrator combines theoretical background with practical exercises to highlight key differences between various anonymity solutions.

A key goal of the demonstrator is to show that there is no single universal ACN suitable for all use cases. Instead, the choice of the network should be guided by the specific requirements of a given scenario. Through hands-on tasks such as clearnet browsing, hidden service setup, and anonymous file sharing, students will gain insight into which networks are better suited for particular applications and why.

This demonstrator was created to show students the practical effects of the comparative analysis from Chapter~\ref{chap:Analysis}, where it was shown that there is no single universal ACN for all use cases. The aim is to let students see in practice how the networks differ and why these trade-offs matter. The choice of networks and exercises reflects the main categories discussed in the analysis. Students are expected to know basic Linux and networking; technical setup is described later. Reading a theoretical introduction to the laboratory is strongly advised. The demonstrator is included here to turn theory into practical skills and highlight the most important findings of the thesis. This chapter first describes the scope of the laboratory, then the course of the laboratory, design considerations followed by the laboratory setup, and finally points to the theoretical introduction and exercises in \nameref{chap:AppendixC}.

\section{Scope of the laboratory}
This laboratory is designed to help students understand how anonymous communication networks (ACNs) work in practice, what their main features are, and where they can be used. The lab covers both the theory and practical exercises to show differences between various anonymity solutions, starting from basic proxies and moving to more advanced systems. Students will learn about the difference between anonymity by policy and anonymity by design and why this matters, the basics of Mix-nets and onion routing and how they affect anonymity, the most popular ACNs today: Tor, I2P, Lokinet, and Nym: How do they work, what they are good at, and what problems they face, as well as the real use cases for these systems, including web browsing, hidden services, and file sharing.

\section{Course of the laboratory}
During the lab, students will go through five main practical exercises:
\begin{enumerate}
    \item Anonymity by policy vs. anonymity by design
    \item Mix-nets and onion routing
    \item Browsing clearnet with ACNs
    \item Hidden Services
    \item File sharing with ACNs
\end{enumerate}

In the first exercise, students will see how single-hop proxies work, check the logs, and notice the weaknesses. Then, they will switch to an ACN setup and compare the difference. After finishing this exercise, students should be able to tell the difference between these approaches, along with the drawbacks and potential threats to anonymity by policy design.

In the second exercise, students will send messages using onion routing and Mix-nets and look at the results. They will see how Mix-nets provide better protection against traffic analysis, but with the cost of a larger delay. After finishing this exercise, students should be able to tell the difference between ACNs based on Mix-nets and onion routing, along with the advantages and disadvantages of each design. Students should also be able to name several use cases for each design.

In the third exercise, students will try to reach regular websites using ACNs, comparing the support of the clearnet for each ACN. After finishing this exercise, students should be able to configure ACNs to browse clearnet. Students should be able to choose a preferred ACN for this application area.

In the fourth exercise, students will set up a Tor hidden service, first without and then with authorisation keys. They’ll see how this allows access to web pages and SSH services securely. After finishing this exercise, students should be able to configure hidden services within the Tor network. A student should be able to name their advantages and potential use cases.

In the last exercise, students will measure the download speed of a large file shared over Tor via OnionShare and I2P via I2PSnark. They will compare which is faster and which hides traffic patterns better. After finishing this exercise, students should be able to point out the differences between Tor and I2P in terms of file sharing. The student should point out a preferable ACN for this use case.

\section{Design considerations}
To maximise educational value within the time constraints of a typical laboratory session (1.5 to 2 hours), the first two exercises use simplified simulators rather than real-world anonymous communication networks. This approach was chosen to more clearly visualize the concepts of anonymity by policy versus anonymity by design, as well as to illustrate the core differences between Mix-nets and onion routing architectures without the overhead of complex setups. Exercises 3 to 5 focus on practical tasks with actual ACNs (Tor, I2P, and Lokinet), chosen for their relevance to real-world applications and their significance in a given category, as proven in the analysis. Not all networks or categories could be included; for example, Nym was omitted due to its commercial distribution model (NymVPN as a paid service), and latency was only compared at the architectural level (Mix-nets vs. onion routing), as a detailed evaluation of minor architectural differences between onion routing-based solutions would be less instructive for the demonstrator’s goals. Similarly, the hidden services exercise is focused exclusively on Tor, as it turned out to be the most favourable solution in the comparative analysis for the infrastructure security and resilience-based use cases. These project decisions ensure that the demonstrator provides focused, high-impact learning experiences that align closely with the findings of the analysis, even if some technical details and alternative solutions were necessarily excluded due to practical considerations.

\section{Laboratory setup}
Laboratory should be performed on a Debian-based Linux distribution, ideally on Debian 12/Bookworm. Each workstation should have Python, Docker, Firefox, any non-Firefox browser (for example, DuckDuckGo browser, Opera or Brave), OnionShare, I2P router, and Lokinet with GUI installed. I2P router and Lokinet should be turned on in order to establish necessary tunnels and paths in advance. I2P's bandwidth parameters should be changed in both router client and I2PSnark client in order to avoid throttling.

\section{Exercises}

Practical exercises, along with the theoretical introduction, are included in \nameref{chap:AppendixC}. Files

\section{Summary}

This chapter presented an educational demonstrator that combines theoretical background and practical exercises to help students understand how anonymous communication networks (ACNs) work in practice. The chapter described the scope of the laboratory, the main exercises and the technical setup, with more details and step-by-step instructions provided in \nameref{chap:AppendixC}.

The main goal of the demonstrator is to show that there is no single universal ACN for all use cases, as established in the comparative analysis in Chapter~\ref{chap:Analysis}. The exercises and their design reflect the most important categories identified in the analysis, allowing students to directly observe the trade-offs and limitations discussed earlier in the thesis. 

The demonstrator can be used as a teaching tool in courses on network security, privacy, or related topics, or as a self-study guide for anyone interested in anonymous communication technologies. By working through the exercises, students gain practical skills, see how theoretical concepts apply in real-world scenarios, and learn to evaluate which ACN is best suited for specific applications. In the context of the thesis, this chapter translates the findings of the multi-criteria analysis into an educational resource, highlighting the practical impact of the research.
