\chapter{Introduction}
\label{chap:Introduction}
This chapter serves as a starting point for the thesis. It defines the motivation for taking up the topic of anonymous communication networks (ACNs) and outlines the main goals of the work. The chapter also specifies the scope of the analysis and presents the structure of the thesis. The intention is to give a clear orientation before moving on to the technical and analytical content that follows. This introduction helps to understand why the subject was chosen, what will be analysed, and how the thesis is organised.

\section{Goal of the work}
Ensuring the anonymity of communication in diverse network systems is a technical challenge that requires solutions capable of achieving this objective while maintaining the usability of the network system. The aim of this work is to analyse the types of technical solutions currently used in anonymising networks/anonymous communication networks (ACNs) in the context of their usability in specific usage scenarios.

\section{Scope of the work}
This work provides an overview of ACNs, emphasising their technical aspects, identifies potential use cases and application areas, categorises these use cases, presents potential threats, and offers a comparative analysis of existing anonymous communication networks in terms of usability in various scenarios. The work also determines the desired directions for the development of ACNs and includes the design and implementation of an educational demonstrator to verify and illustrate key elements of the analysis.

\section{Structure of the work}
This work begins with \nameref{chap:Introduction}, which presents the objectives and structure of the thesis.
The second chapter is \nameref{chap:Theory}, which explains all the theoretical concepts necessary to understand the paper.
The third chapter is \nameref{chap:Overview}. It provides an overview of the most prominent ACNs is provided, focusing on those most popular today and those that have significantly influenced current solutions.
Then, in the fourth chapter \nameref{chap:Related}, the related work on the analysis of anonymous communication networks is presented.
Following the related work, in the fifth chapter \nameref{chap:UseCases}, the use cases for ACNs are proposed. The use cases are then organised into groups of similar requirements. The criteria for each group are selected and appropriately weighted.
After presenting the use cases, in the sixth chapter \nameref{chap:Threats} the relevant threats are discussed, including limitations and possible attacks.
Subsequently, in the seventh chapter \nameref{chap:Analysis}, the technical comparison of the most popular ACNs and their technical solutions is carried out in two stages. The first stage is the comparison based solely on the literature. Then, the experiments are performed and the results are used in the second stage, an empirical and experimental comparison.
After comparisons, ACNs are evaluated in terms of usability for the identified use case groups and their requirements. Based on this evaluation, the best-suited ACN is identified for each use case group.
Following the analysis, in the eighth chapter \nameref{chap:Future}, the future directions of ACNs are described.
After presenting future directions, in the ninth chapter \nameref{chap:EducationalDemonstrator}, an educational demonstrator is designed to verify and illustrate the ACN analysis.
Finally, the work concludes with \nameref{chap:Summary} a summary of the findings and a discussion of the results.
There are also appendices; \nameref{chap:AppendixA} provides scripts for setting up the services that will be used to perform measurements, \nameref{chap:AppendixB} provides the application used for measurements, \nameref{chap:AppendixC} provides the laboratory instruction from the educational demonstrator along with the theory introduction for the laboratory, and \nameref{chap:AppendixD} includes the code for the simulator that is used within the educational demonstrator.

\section{Summary}
In summary, this chapter sets the foundation for the thesis by clearly stating its objectives, establishing the boundaries of the analysis, and presenting the overall structure that will be followed in the next chapters. Setting these foundations is crucial for presenting the reasoning behind the work and to know what to expect in the thesis. The introduction clarifies the function of each part of the thesis and shows how the analysis will be approached in the following chapters.
