\chapter*{Abstract}

This thesis presents an analysis of anonymous communication networks (ACNs), focusing on their technical foundations, practical applications, and security properties. The work begins with a theoretical overview of the evolution of ACNs, distinguishing between anonymity by policy and anonymity by design, and examining key architectural models such as Mix-nets and onion routing.


A central part of the thesis is a multi-criteria comparative analysis, combining literature review with experimental measurements of ACN usability and effectiveness across different use case scenarios, performed for today's ACNs: Tor, I2P, Lokinet, and Nym. The results clearly demonstrate that no single ACN is universally optimal; instead, the best choice depends on the specific use case.


To support the practical understanding of these findings, an educational demonstrator was developed. This demonstrator enables students to explore ACN designs in hands-on laboratory exercises, observe their strengths and weaknesses, and understand how different approaches mitigate specific vulnerabilities. The work concludes with recommendations for future research, including monitoring new ACN designs, conducting advanced performance measurements under varied configurations, and studying the impact of network parameters on usability and security.


Keywords: Tor, I2P, Loki, Nym, ACN, analysis, use case, application area, usage
