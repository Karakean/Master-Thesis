\chapter{Future directions}
This chapter outlines the future directions recommended for the development of anonymous communication networks. It begins with improvements common to all ACNs, before addressing specific needs and development goals for Tor, I2P, Lokinet, and Nym. These proposals focus on improving performance, usability, security, and network scalability based on current limitations and observations.

\section{Common}
Certain desired future directions are common for each existing ACN.
\begin{enumerate}
    \item Increase the number of users: The popularity of a given ACN directly influences the size of the anonymity set and therefore the provided level of anonymity. The more users the network has, the more diverse they are, the harder it is to target one specific user as he blends into the crowd. In order for the popularity to increase, awareness needs to be spread. One way to do this is by giving talks and interviews. Although people care about their privacy, awareness about privacy by design solutions leaves room for improvement.
    \item Post-quantum cryptography: Quantum computers pose a threat to ACNs that do not use quantum computers-proof cryptography algorithms. The process of introducing them in ACNs is already ongoing, for example, I2P and Nym publicly stated switching to PQC as a part of their roadmap.
    \item Fighting censorship: One of the most important things ACNs need to address is progressive blocking by restrictive countries. In order to address this issue in Tor, bridges were created, but with time they are also getting blocked in one way or another, which was described before in this section. This leads to an unfair arm race with several countries that have a huge amount of money they can spend on ACN censorship. Current bridge distribution mechanisms seem to be insufficient, and pluggable transports were introduced that will further enhance the censorship circumvention capabilities of Tor.
    
    Although other ACNs such as Lokinet and Nym face similar blocking, they have not yet implemented mitigation techniques. They can utilise similar mechanisms of bridges and pluggable transports as Tor does or introduce custom censorship circumvention solutions. Although I2P seems to be the least censored ACN, most likely due to the peer-to-peer architecture and large number of routers in general, it can still be censored via DPI. Maybe censoring regimes consider I2P as a niche solution and therefore unworthy of censoring, although once the network gets more recognition, the more aggressive blocking will likely be imposed. I2P should therefore also propose censorship circumvention techniques.
\end{enumerate}

\section{Tor}
\begin{enumerate}
    \item Tor Browser development: One of the main features of anonymous communication networks is the number of users, as it directly influences anonymity. In addition to the number of them, the users should be as diverse as possible. In order to encourage more people to use Tor, the software needs to be as user-friendly as possible. A great deal of progress has been made in this field over the years, resulting in an easy-to-use Tor Browser. Sadly, it is not yet available on iOS devices. Alternatives like Orbot need to be utilised; however, it lacks the security properties of the Tor Browser.
    \item Fighting with blocking exit nodes on websites: Certain websites block traffic that originates from Tor or discriminate against Tor users by other means. An example of a partial discrimination in this field is present in Wikipedia where Tor users can view articles but cannot edit them, despite the fact that they are logged into their accounts or not.
    \item Decreasing delays: Although the performance of the relays in terms of committed bandwidth is dependent on the runners of the relays, the delays can be improved on the Tor project side. Potentially ongoing rewrite to Rust may help with it as Rust can better utilise concurrency. In theory, cryptography could also be improved - it should be examined whether pre-computing approaches as described with cMix would improve Tor’s performance. Tor could also include UDP support or move down one layer and work within the Network layer. As UDP is becoming increasingly popular in the web thanks to QUIC, enabling for faster web browsing, supporting it can be beneficial in terms of end-user delay. However, supporting UDP would be a major change.
\end{enumerate}

\section{I2P}
\begin{enumerate}
    \item Enhancing Sybil attack resistance: For a long time, I2P did not have any solutions to address Sybil attacks. The work on it is ongoing, and there is a mechanism for detecting such an attack via IP closeness. Sadly, it is not yet sufficient for a full mitigation of this attack. One of the proposals was to include proof-of-work mechanisms for creating a new identity, making the attack more expensive to perform.
    \item Decreasing latency: As was shown with the latency measurements, I2P suffers from a significant latency. I2P included speeding up cryptography as a part of their roadmap, which will have a positive impact on latency. As part of speeding up, they can consider precomputation proposals as in Chaum’s cMix. Another aspect that may influence latency is the complexity of the I2P protocol stack - a simplification of it may be considered if decreasing latency by other means does not help.
    \item Improving UDP tunnels support: Mainline I2P does not support basic UDP tunnels yet - it is only possible with I2PD - a C++ implementation of an I2P router. Even then the integration is not perfect; it was noticed during the measurements that while iperf3 worked fine with TCP tunnels, it did not work with UDP ones.
\end{enumerate}

\section{Lokinet}
\begin{enumerate}
    \item Improving documentation: The Lokinet documentation is relatively modest compared to Tor or I2P. The Lokinet whitepaper puts a significant focus on the token itself, leaving many ACN-related topics undescribed. Even such a basic aspect as the number of hops is not properly explained in the documentation.
    \item Decreasing latency: Lokinet is heavily focused on latency. It should be examined whether the number of hops is not too large. Although Lokinet has indeed lower latency than Tor, the difference is not that significant, as was shown in the measurements results. Potentially, there is also room for improvement within the cryptography used.
    \item Increasing number of nodes: Despite the fact that the number of nodes in Lokinet is not the lowest, it still should be increased. Currently, running nodes is associated with a significant cost. While it may be beneficial in terms of Sybil attack resistance, it may discourage some node runners who would like to voluntarily run nodes rather than utilise a for-profit token scheme.
\end{enumerate}

\section{Nym}
\begin{enumerate}
    \item Increasing number of nodes: As Nym is the youngest network and was just recently officially launched, it has the lowest number of nodes within the network. Although the placement of nodes seems to be diversified, the number could have been increased to make the network safer, in general.
    \item Modifying the level provided on anonymity: While both Nym and Loopix paper describe modifying network parameters in a way that would allow for an adjustment of provided anonymity level and latency, it is not modifiable yet.
    \item Improve documentation: Despite the fact that Nym supports a communication fully inside the network, it is not yet clear fully how to set up such a communication, and there were documentation changes recently. It would be beneficial if, for example, a whistleblowing platform could have easily set up a service that allows users to access it without leaving the network and, at the same time, increases user anonymity as the service receives cover traffic. Nym could also have provided a simple list of available services within the network if the service provider wishes for his service to be publicly available.
\end{enumerate}
