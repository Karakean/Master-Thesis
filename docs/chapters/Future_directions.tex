\chapter{Future directions}
\label{chap:Future}

Previous chapters, especially the comparative analysis, points to the application areas where ACNs are especially useful. In order to maintain this usability and develop it further, appropriate steps should be taken. The aim here is to show not only what needs fixing, but also where the most significant improvements can realistically be made. The structure is simple: first, general recommendations that every ACN can benefit from, then dedicated sections for each network, focusing on the specific issues and future priorities that emerged from earlier analysis. It gives a clear direction for the desired development directions.

\section{Common}
This section addresses future directions that are relevant to all today's ACNs, regardless of architectural or protocol differences. The previous chapters highlighted several challenges, such as the need of a high user adoption, the potential threat of quantum computing, and the ongoing arms race with censors, that is affecting or can potentially affect every network.

The popularity of a given ACN directly influences the size of the anonymity set and therefore the provided level of anonymity. The more users the network has, the more diverse they are, the harder it is to target one specific user as he blends into the crowd. In order for the popularity to increase, awareness needs to be spread. One way to do this is by giving talks and interviews. Although people care about their privacy, awareness about privacy by design solutions leaves room for improvement.

Another crucial consideration in the future of ACNs is the support for post-quantum cryptography. Quantum computers pose a threat to ACNs that do not use quantum computers-proof cryptography algorithms. The process of introducing them in ACNs is already ongoing, for example, I2P and Nym publicly stated switching to PQC as a part of their roadmap.

One of the most important things ACNs need to address is progressive blocking by restrictive countries. In order to address this issue in Tor, bridges were created, but with time they are also getting blocked in one way or another, which was described before in this section. This leads to an unfair arm race with several countries that have a huge amount of money they can spend on ACN censorship. Current bridge distribution mechanisms seem to be insufficient, and pluggable transports were introduced that will further enhance the censorship circumvention capabilities of Tor.
Although other ACNs such as Lokinet and Nym face similar blocking, they have not yet implemented mitigation techniques. They can utilise similar mechanisms of bridges and pluggable transports as Tor does or introduce custom censorship circumvention solutions. Although I2P seems to be the least censored ACN, most likely due to the peer-to-peer architecture and large number of routers in general, it can still be censored via DPI. Maybe censoring regimes consider I2P as a niche solution and therefore unworthy of censoring, although once the network gets more recognition, the more aggressive blocking will likely be imposed. I2P should therefore also propose censorship circumvention techniques.

\section{Tor}

Following from the previous analysis, this section focuses on future work for Tor, especially in those use cases where it performed best and in areas where limitations were most apparent. Specific attention is given to improving usability, broadening platform support, addressing discrimination by online services, and exploring deeper technical enhancements that could sustain or even expand Tor’s strengths.

In the comparative analysis, Tor was shown to be the most suitable ACN for web browsing-based use cases, thanks to its strong clearnet support, robust censorship resistance, low latency, and user-friendly tools like the Tor Browser. To further enhance Tor’s advantage in this area, future improvements should focus on optimising web browsing experience even further by speeding up page loads, improving exit node reliability, and making setup seamless on all major platforms, including mobile. Addressing these areas will help Tor remain the de facto choice for privacy-conscious web users while keeping pace with the evolving expectations of everyday Internet browsing.

One of the main features of anonymous communication networks is the number of users, as it directly influences anonymity. In addition to the number of them, the users should be as diverse as possible. In order to encourage more people to use Tor, the software needs to be as user-friendly as possible. A great deal of progress has been made in this field over the years, resulting in an easy-to-use Tor Browser. Sadly, it is not yet available on iOS devices. Alternatives like Orbot need to be utilised; however, it lacks the security properties of the Tor Browser. Tor Browser development is significant as Tor is the most suitable network for web browsing-based activities, as was proved in the comparative analysis chapter.

Certain websites block traffic that originates from Tor or discriminate against Tor users by other means. An example of a partial discrimination in this field is present in Wikipedia where Tor users can view articles but cannot edit them, despite the fact that they are logged into their accounts or not. There is a need to make the website runners aware of the legitimate intentions of most of the ACN users.

Although the performance of the relays in terms of committed bandwidth is dependent on the runners of the relays, the delays can be improved on the Tor project side. Potentially ongoing rewrite to Rust may help with it as Rust can better utilise concurrency. In theory, cryptography could also be improved - it should be examined whether pre-computing approaches as described with cMix would improve Tor’s performance. Tor could also include UDP support or move down one layer and work within the Network layer. As UDP is becoming increasingly popular in the web thanks to QUIC, enabling faster web browsing, supporting it can be beneficial in terms of end-user delay. However, supporting UDP would be a major change.

Tor also turned out to be the most suitable choice in the infrastructure security and resilience-based use case group, as confirmed in the comparative analysis chapter. This is largely due to its established authorisation mechanisms, resilience to active attacks, reasonable server anonymity, and the maturity and depth of academic research behind the project. In terms of future improvements, it will be crucial for Tor to further strengthen relay security and to encourage node runners to maintain a diverse and globally distributed network of relays, something that directly impacts its resistance to both technical and organisational attacks and lowers the chance of a powerful adversary to control enough nodes to successfully perform complex passive attacks. At the same time, Tor Project should continue investing in research to proactively identify and address emerging threats.

\section{I2P}

This section looks at the specific recommendations for I2P, in light of its performance in different categories during the comparative analysis and risks defined in previous chapters. Issues like Sybil resistance, protocol latency, and incomplete UDP support proved to be real-world bottlenecks. The focus should be put on making I2P more robust and competitive, building on its existing strengths while addressing its weakest points.

The comparative analysis clearly demonstrated that I2P is best suited for file sharing-based use cases, primarily due to its peer-to-peer architecture, strong support for file sharing protocols like BitTorrent, and robust volume correlation resistance. To further reinforce I2P’s position as the top ACN for file sharing, improvements should be directed towards optimising peer discovery, increasing the speed and reliability of large downloads and providing additional size obfuscation methods. Optional cover traffic is a proposition that should be examined. Very trivial potential improvement to file speed is to change the default bandwidth settings - current ones are significantly low and there is a high possibility that many users may not adjust their bandwidth to their capabilities.

For a long time, I2P did not have any solutions to address Sybil attacks and it is one of the biggest threat to the I2P network right now, as was described in the threats chapter. The work on it is ongoing, and there is a mechanism for detecting such an attack via IP closeness. Sadly, it is not yet sufficient for a full mitigation of this attack. One of the proposals was to include proof-of-work mechanisms for creating a new identity, making the attack more expensive to perform. Nevertheless, enhancing Sybil attacks, described more extensively in the dedicated threats and attacks chapter, resistance should be a crucial consideration.

As was shown with the latency measurements, I2P suffers from a significant latency. I2P included speeding up cryptography as a part of their roadmap, which will have a positive impact on latency. As part of speeding up, they can consider precomputation proposals as in Chaum’s cMix. Another aspect that may influence latency is the complexity of the I2P protocol stack - a simplification of it may be considered if decreasing latency by other means does not help.

Mainline I2P does not support basic UDP tunnels yet - it is only possible with I2PD - a C++ implementation of an I2P router. Even then the integration is not perfect; it was noticed during the measurements that while iperf3 worked fine with TCP tunnels, it did not work with UDP ones. There is significant room for improvement for UDP support. Once a decent UDP support is provided, the potential I2P's use cases might be wider than they are now.


\section{Lokinet}

This section outlines key areas for Lokinet’s future growth. In the comparative analysis, Lokinet stood out as the most effective solution for low-latency inter-network communication use cases, thanks to its consistently low latency, low jitter, and high throughput along with support for UDP traffic. The focus is on improving transparency and usability through better documentation, re-evaluating current architectural choices, and increasing the number and diversity of nodes.

Lokinet is heavily focused on latency. It should be examined whether the number of hops is not too large - according to the documentation, it includes 7 hops, while Tor utilises 6 for hidden services, I2P as well. Although Lokinet has indeed lower latency than Tor, the difference is not that significant, as was shown in the measurements results. Potentially, there is also room for improvement within the cryptography used. Ideally, while aiming the latency decrease, the jitter decrease should be addressed as well - potentially it will decrease alongside the latency. 

High throughput is yet another advantage of Lokinet. Throughput can be increased by providing even larger rewards for node runners that contribute with a significant bandwidth.

The Lokinet documentation is relatively modest compared to Tor or I2P. The Lokinet whitepaper puts a significant focus on the token itself, leaving many ACN-related topics undescribed. Even such a basic aspect as the number of hops is not properly explained in the documentation. The documentation could definitely be improved.

Despite the fact that the number of nodes in Lokinet is not the lowest, it still should be increased. Currently, running nodes is associated with a significant cost. While it may be beneficial in terms of Sybil attack resistance, it may discourage some node runners who would like to voluntarily run nodes rather than utilise a for-profit token scheme.


\section{Nym}

Nym, as the youngest ACN in this comparison, stands out in the high anonymity, latency-tolerant use case group. Its Mix-net design and cover traffic make it much harder to attack with traffic analysis, giving it a clear advantage here. Still, there is room for improvement: the number of nodes could be higher, parameter adjustments promised in the literature are not yet available, and the documentation around intra-network services could be clearer. Addressing these would further strengthen Nym’s position as the go-to choice for users prioritising maximum anonymity.

As Nym is the youngest network and was just recently officially launched, it has the lowest number of nodes within the network. Although the placement of nodes seems to be diversified, the number could have been increased to make the network safer, in general.

While both Nym and Loopix paper describe modifying network parameters in a way that would allow for an adjustment of provided anonymity level and latency, it is not modifiable yet. With this mechanism implemented, Nym can provide even higher anonymity than it does now, further enhancing its advantage in the category of high anonymity and latency-tolerant use cases.

Despite the fact that Nym supports a communication fully inside the network, it is not yet clear fully how to set up such a communication, and there were documentation changes recently. It would be beneficial if, for example, a whistleblowing platform could have easily set up a service that allows users to access it without leaving the network, providing built-in storage and, at the same time, increases user anonymity as the service receives cover traffic. Nym could also have provided a simple list of available services within the network if the service provider wishes for his service to be publicly available.

\section{Summary}

This chapter reviewed future directions for ACNs, based on the previous chapters, especially the comparative analysis. The key common points, like growing the user base, getting ready for quantum-safe cryptography, and pushing back against increasing censorship, apply to every ACN, no matter the architecture. Then, each network got its own list of next steps, tailored to the area where it currently leads: web browsing along with infrastructure security and resilience for Tor, file sharing for I2P, low-latency for Lokinet, and strong anonymity for Nym. Each of these sections ties directly back to these ACNs' limitations and strengths. In the broader context of the thesis, this chapter points out the practical steps that need to be taken next, if the aim is to actually build safer and more usable anonymous communication networks.
