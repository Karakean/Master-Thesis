\chapter*{Appendix B: Python application for measuring RTT and jitter}
\label{chap:AppendixB}
\addcontentsline{toc}{chapter}{Appendix B: Python application for measuring RTT and jitter}

\subsubsection{Server}

The following script is the server component of the Python application to measure RTT and jitter. It listens on port 2135 for incoming TCP connections and handles client requests by echoing received data back to the sender. The server operates in a continuous loop, handling one client connection at a time.

\footnotesize
\begin{lstlisting}[language=python, breaklines=true, breakatwhitespace=true, showstringspaces=false]
import socket

HOST = '0.0.0.0'
PORT = 2135
BUFFER_SIZE = 1024

def run_server() -> None:
    server_socket = socket.socket(socket.AF_INET, socket.SOCK_STREAM)
    server_socket.bind((HOST, PORT))
    server_socket.listen()
    print(f"TCP latency server is running on port {PORT}...")
    while True:
        client_socket, client_addr = server_socket.accept()
        print(f"Connection from {client_addr}")
        handle_client(client_socket, client_addr)

def handle_client(client_socket, client_addr) -> None:
    try:
        while True:
            data = client_socket.recv(BUFFER_SIZE)
            if not data:
                break
            client_socket.sendall(data)
    except Exception as e:
        print(f"Error: {e}")
    finally:
        client_socket.close()
        print(f"Connection with {client_addr} closed")

if __name__ == "__main__":
    run_server()
\end{lstlisting}
\normalsize

\subsubsection{Client}

This script is the client component of the Python application. It connects to the server (localhost for I2P client tunnel, it should be replaced with appropriate onion address for Tor or loki address for Lokinet) on port 2135 and performs a series of latency measurements by sending timestamped messages and calculating the round-trip time (RTT) for each. The script computes and displays the average RTT, jitter, and packet loss statistics. Although this specific version of the application is designed for reliable in-order delivery via TCP, the client is implemented to handle the potential out-of-order messages. This design choice makes it easy to adapt the script for UDP-based communication, requiring only minor adjustments to the socket type and the socket-related communication logic.

\footnotesize
\begin{lstlisting}[language=python, breaklines=true, breakatwhitespace=true, showstringspaces=false]
import socket
import time
import statistics

HOST = "127.0.0.1"
PORT = 2135
TIMEOUT = 10
TRIALS = 50

def run_client() -> None:
    time_pairs = {}  # key: send_time (float), value: recv_time or None
    try:
        client_socket = establish_tcp_socket()
    except Exception as e:
        print(f"Failed to connect: {e}")
        return
    perform_measurements(time_pairs, client_socket)
    client_socket.close()
    avg_rtt, jitter, packet_loss = calculate_results(time_pairs)
    print_results(avg_rtt, jitter, packet_loss)

def establish_tcp_socket() -> socket.socket:
    client_socket = socket.socket(socket.AF_INET, socket.SOCK_STREAM)
    client_socket.settimeout(TIMEOUT)
    client_socket.connect((HOST, PORT))
    return client_socket

def perform_measurements(time_pairs, client_socket) -> None:
    for _ in range(TRIALS):
        send_time = time.time()
        send_bytes = str(send_time).encode()
        time_pairs[send_time] = None
        try:
            client_socket.sendall(send_bytes)
            data = client_socket.recv(1024)
            recv_time = time.time()
            try:
                echoed_send_time = float(data.decode())
            except ValueError:
                print("Received malformed response.")
                continue
            if echoed_send_time in time_pairs and time_pairs[echoed_send_time] is None:
                time_pairs[echoed_send_time] = recv_time
                rtt = recv_time - echoed_send_time
                print(f"RTT = {rtt:.6f} sec")
            else:
                print("Mismatched or duplicate response.")
        except socket.timeout:
            print("Timeout occurred.")
        except Exception as e:
            print(f"Communication error: {e}")
        time.sleep(1)

def calculate_results(time_pairs) -> tuple:
    rtts = [
        recv - sent
        for sent, recv in time_pairs.items()
        if recv is not None
    ]
    avg_rtt = sum(rtts) / len(rtts) if rtts else 0
    jitter = statistics.stdev(rtts) if len(rtts) > 1 else 0
    packet_loss = ((TRIALS - len(rtts)) / TRIALS) * 100
    return avg_rtt,jitter,packet_loss

def print_results(avg_rtt, jitter, packet_loss) -> None:
    print("\n--- Results ---")
    print(f"Average Latency: {avg_rtt * 1000:.2f} ms")
    print(f"Jitter: {jitter * 1000:.2f} ms")
    print(f"Packet Loss: {packet_loss:.2f}%")

if __name__ == "__main__":
    run_client()

\end{lstlisting}
\normalsize