\chapter{Related work}
\label{chap:Related}

This chapter reviews the most relevant related work on anonymous communication networks in terms of comparative studies and practical analyses. The goal is to provide a clear overview of how previous research has approached the analysis and comparison of ACNs, which topics have been covered in depth, and where significant gaps remain, especially in terms of real-world usage scenarios and the inclusion of modern systems.

Including this section is essential to show the current state of research in this field. By understanding what has already been analysed and where the literature falls short, it becomes possible to justify the need for the new, scenario-driven comparative analysis presented in later chapters. The structure of this chapter is straightforward: it overviews existing literature and highlights why an updated and use-case-oriented comparison is still needed.

\section{Overview}

The existing literature on the analysis of anonymous communication networks often concentrates on traffic classification techniques \cite{analysis-classification-jap, analysis-hierarchical-traffic-classification-jap, analysis-packet-momentum-identification-jap, analysis-phd-thesis-identification-jap}, rather than practical usage in various use cases or application areas. Although some studies provide comparative analyses of anonymous networks, they often lack a clearly defined goal or context for their comparison \cite{analysis-tor-vs-i2p}. Other works address usage aspects \cite{analysis-tools-usage-jap, analysis-acn-usage-jap-old}, but these papers provide only general guidance rather than conducting a comparative analysis based on specific use case scenarios.

Furthermore, most of these papers include analysis of legacy systems such as JAP and do not compare more recent network designs such as Lokinet or Nym. As ACNs evolve, older studies become less representative in terms of current performance or development status.

\section{Summary}

In summary, although the literature provides valuable technical analysis of anonymous communication networks, it primarily focuses on aspects like traffic classification and tends to overlook practical, scenario-driven comparisons that reflect how these systems are actually used. With the introduction of new networks such as Lokinet and Nym and the decline of legacy solutions, there is a clear need for up-to-date comparative analysis that considers concrete usage scenarios. Without this perspective, users might not realise that they are not using the best tool for their specific needs, which can lead to suboptimal choices and missed opportunities to improve privacy or efficiency. For these reasons, conducting a comparative analysis rooted in usage scenarios is crucial, not only to fill a gap in the research but also to help users make better informed decisions when choosing an anonymous communication network for a given use case.
