\chapter{Related work}

Existing literature on the analysis of anonymous communication networks includes legacy systems such as JAP and does not compare more recent network designs like Lokinet or Nym \cite{analysis-acn-usage-jap-old, analysis-tools-usage-jap, analysis-classification-jap, analysis-phd-thesis-identification-jap, analysis-packet-momentum-identification-jap, analysis-hierarchical-traffic-classification-jap}. As ACNs evolve, older measurement studies become less representative of the current performance and other properties.

Furthermore, much of the research concentrates on traffic classification techniques \cite{analysis-classification-jap, analysis-hierarchical-traffic-classification-jap, analysis-packet-momentum-identification-jap, analysis-phd-thesis-identification-jap}, rather than practical usage in various use cases or application areas. While some studies provide comparative analyses of anonymous networks, they often lack a clearly defined goal or context for their comparison \cite{analysis-tor-vs-i2p}. Other works address usage aspects \cite{analysis-tools-usage-jap, analysis-acn-usage-jap-old}, but these are typically outdated, covering deprecated ACNs, and do not consider the latest developments in the field. Moreover, mentioned papers provide only general guidance rather than conducting a comparative analysis based on specific use case scenarios.
