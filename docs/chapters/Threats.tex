\chapter{Threats, attacks and limitations}


\section{Attacks}

\subsection{Passive}
Passive attacks in ACNs are mostly related to traffic analysis. Detecting this category of attack is almost impossible.
\begin{enumerate}
    \item Timing correlation: One of the most effective attacks against low latency ACNs. Given an adversary who can watch traffic entering and leaving the network, he can easily distinguish who sent the message due to the timing correlation. In order to mitigate this kind of attack, a non-deterministic strategy needs to be included. In Mix-nets, batching with various strategies is deployed. Addressing the timing correlation is always associated with additional latency and therefore smaller usability in lower-latency usage scenarios. While there are propositions of optional inclusion of non-trivial delays and batching strategies in I2P, they are not yet implemented. As in I2P each user acts as a router it will impose additional effort on the attacker to distinguish the traffic originating from the user from the traffic he is passing through, however it is not a complete mitigation of this type of an attack. Tor and Lokinet are the most vulnerable to the timing correlation attack, however the more users the network has, and therefore the larger anonymity set is, the harder it is to perform this attack.
    \item Size correlation: Attack in which an adversary correlates the message with its sender due to its size. Lokinet is the most vulnerable to this type of attack as it does not propose any mitigation. Tor utilises fixed-size messages, however it is only a partial mitigation of the issue, especially when a user is sending large messages, as the sequential chunks clearly form a larger message and the pattern can be therefore observed. In I2P the chunks can be initially combined into larger messages and then, after leaving the sender’s outbound tunnel, split and delivered through multiple receiver’s inbound tunnels, which can make the size correlation harder to perform. Nym provides the best defense against this attack, providing both padded Sphinx messages along with cover traffic which makes observing message size patterns impossible.
    \item Intersection attack/statistical disclosure/traffic pattern observation: Identifying users by correlating online presence over time. When an adversary can observe one endpoint of the communication as well as the fact of a user entering the network, the fact of intensified message number on the receiving side can be associated with a certain user that is most likely a sender. This attack is especially feasible when an adversary can observe the network for a long period of time. In theory, when a user is constantly connected to the ACN, it would not be distinguishable, however this is not an acceptable assumption to make. The more users and routers are in the network, the more complex the attack is to be performed. Masking traffic and non-public routers, for example pluggable transports and bridges from Tor network, can also mask a fact of connecting to the network. The best solution to this issue, deployed by Nym, is to include a cover traffic - as every endpoint is constantly receiving it, an adversary is not able to tell whether a certain entity is receiving larger traffic than usual. Batching and mixing is yet another potential solution to this issue.
    \item Browser fingerprinting: browser-related attack that involves collection of user’s data that can be later used to identify him. Tor solves this issue by providing a custom, hardened browser. While other ACNs are vulnerable to this issue, it is less performable in hidden services-focused ACNs like I2P.
    \item Unencrypted content observation: an attack associated with malicious exit nodes that can observe unencrypted traffic of users that are reaching clearnet resources over unsafe protocols such as HTTP. The solution for this attack is to either only use end-to-end encrypted protocols, for example, HTTPS-based websites, or not to leave the ACN at all and focus on the hidden services.
\end{enumerate}

\subsection{Active}
\begin{enumerate}
    \item Key exposure: while brute force-based attacks on cryptographic keys is not feasible in acceptable time with the current technology, the key can be acquired by other means. In case of keys that are not periodically rotated, and therefore ACNs that do not provide perfect forward secrecy, it may be potentially used to decrypt an encryption layer from the past. In most of the ACNs, the keys are ephemeral, and therefore leaked keys can only be used to unwrap one layer of encryption for a short period of time.
    \item Tagging attack: attack based on modifying a message in a way that it would be distinguishable later in the message path. Message integrity checking in ACNs is crucial when it comes to defending against this type of attack and most of them include the mechanisms for providing integrity.
    \item Content modification: an attack that is one step further compared to the unencrypted content observation passive attack described above. In this case the malicious exit node modifies content before passing it through. The mitigation is the same as described with the unencrypted content observation attack.
    \item Sybil attack: an attack in which an adversary creates a large number of routers and tries to, for example, deanonymise users or perform other attacks such as DoS. I2P is particularly vulnerable to this attack due to its peer-to-peer architecture and the lack of effective mitigation mechanisms. In Tor the attack needs to be performed over a long period of time as a sudden spike in the number of nodes can be easily noticed and addressed. Lokinet and Nym consider themselves as more resistant to this type of attack due to economic incentives for the node runners. Nym also includes penalisation of the operators that register multiple nodes.
    \item N - 1 attack: an attack aimed for Mix-nets in which an adversary floods the mix with his n - 1 messages, leaving space for one non-adversary message which he aims to target. Nym addresses this issue by using cover traffic and stratified topology.
    \item Partitioning attack and eclipse attack: an attack in which an adversary tries to isolate a single target (eclipse attack) or a larger network segment (partitioning attack), forcing victims to connect to the routers controlled by the adversary. The attack involves interference into the mechanisms of discovering routers within the network, for example netDb in I2P or directory servers in Tor. ACNs such as I2P with more decentralised nature during new user joining, can also suffer from a specific type of the eclipse attack called the bootstrap attack in which an attacker provides a list of his malicious routers instead of the legitimate ones.
    \item Denial of Service (DoS): attack that aims to disrupt an ACN. The attack has its distributed version - DDoS, that is performed from many places. The larger the network is, the more costly the attack becomes. In peer-to-peer ACNs like I2P naive DDoS may potentially have the opposite effect as each new user will also act as a router and speed-up the network, unless the attacker modifies his nodes to only consume from the network and do not contribute. There are many forms of DoS, often addressing ACN-specific technical solutions, for example Sniper attack \cite{sniper} on Tor.
    \item Compulsion attack: non-technical type of attack in which a router runner is forced to give up control over his router in order for the adversary to control the traffic passing through it. This attack is not useful against the networks that have perfect forward secrecy, meaning the frequent key rotation, as gaining control over nodes is not useful for decrypting previous messages. While this attack may have been more dangerous if the node runner ran all nodes in a given path, most of the ACNs provide diversification mechanisms for the nodes in order to avoid them from being under the same jurisdiction et cetera.
    \item Cryptographic attacks: while today's ACNs use algorithms considered safe, meaning that they cannot be brute forced in an acceptable time, the highest danger is associated with quantum computers and their ability to break currently safe cryptographic algorithms. While it is not a threat that is present today, Tor and I2P are already in the process of replacing their currently used algorithms with post-quantum-based ones.
    \item Bugs: attacks that exploit code vulnerabilities are mostly related to less mature ACNs with less research backup.
    \item ACN-specific attacks: there are many ACN-specific attacks that are usually an effect of omitting a certain aspect during projecting of the network. Contrary to bugs, in this example the ACN works as design but the issue lies with the design itself. The solution is also more costly than a resolution of a bug as it involves changes in the architecture. An example can be guard discovery attack \cite{guard-discovery} on Tor.
\end{enumerate}

\section{Economic sustainability}
Economic sustainability is a critical challenge in ACNs. The approach to the economic issues related to ACNs varies from network to network. Peer-to-peer solutions, like I2P, benefit from the decentralized nature. Each I2P user that utilises the network contributes to it for the others, enhancing scalability and anonymity. The development team behind I2P does not accept direct donations; they encourage supporting application developers instead. One of the examples is StormyCloud that provides a free outproxy service for I2P.

Tor has an entirely altruistic volunteer-based model in terms of running nodes. This approach has turned out to be working well for over two decades of Tor history. Tor underlines the importance of diversity among their nodes, and the approach they have chosen seems to align with this goal.

Lokinet focuses on economic incentives for node runners, rather than volunteer-based networks. While this approach can attract a broad set of participants, which it currently does, it introduces several potential challenges. Firstly, the node runners may choose the cheapest infrastructure providers in order to maximise their profit, which may lead to lesser diversity of nodes within the network. Secondly, the token-based reward that Lokinet proposes may be volatile. The value of the token they provide is strictly speculative and may vary significantly over time, potentially discouraging node runners when the price goes down. The development team does not accept direct donations; similarly, to the node runners, they benefit from the tokens.

The Nym network has a similar approach to Lokinet in terms of incentives for node runners. While this can encourage more node runners, it shares the same risks, including token volatility and centralisation issues. Contrary to other ACNs, Nym requires a monthly fee for the network usage that is funding the development behind Nym.

\section{Censorship arms race}
Efforts to circumvent Internet censorship using ACNs are continually met with countermoves by restrictive governments, creating an ongoing technological arms race. Tor is in particular an object of such actions \cite{tor-arms-race} as it is the most popular ACN today. Governments try to block ACNs by targeting their management mechanisms, public relay nodes for client-server ACNs, websites from which you can download a given ACN, or by using deep packet inspection (DPI) to detect ACN-specific traffic. In response, the Tor Project has developed solutions such as bridge relays and pluggable transports to help users access the network even under heavy censorship. While censors have access to significant resources and increasingly sophisticated techniques, the Tor Project adapts by introducing new technical measures, making censorship circumvention a continuous and evolving challenge. Other ACNs have not yet deployed such advanced censorship circumvention techniques; although it is often a part of their roadmap.
