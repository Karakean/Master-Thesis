\chapter{Summary}
\label{chap:Summary}
The main goal of this thesis was to determine whether there is a universal anonymous communication network exists; one that is optimal for all use case scenarios. The analysis conducted here, based on both the technical literature and the experimental results, leaves no room for doubt: there is no such universal solution. The choice of an anonymous communication network (ACN) always comes down to the specific requirements and threat model of the intended application.

The thesis began by outlining the background and motivation for studying ACN in the \nameref{chap:Introduction} chapter, defining the scope of the work, and explaining why this topic is relevant. 

The \nameref{chap:Theory} chapter established the necessary knowledge foundations, necessary to understand the insights, strengths, and trade-offs of anonymous communication systems.

Next, the \nameref{chap:Overview} chapter presented a detailed overview of the major types of ACNs in use today, introducing the main architectural approaches, such as single-hop proxies, Mix-nets, DC-nets, anonymous publication systems, and onion routing, offering a concise description of the networks that were analysed in detail later in the thesis. This chapter set the scene for a deeper analysis, ensuring that subsequent comparisons were provided with an appropriate context.

The \nameref{chap:Related} chapter provided a critical overview of existing research, highlighting the lack of practical, use case-driven comparisons between major networks.

The chapter \nameref{chap:UseCases} mapped out the real-world use cases and application areas of ACNs, categorised them according to their technical requirements, and established clear criteria for evaluation. This classification formed the basis for the subsequent analysis and ensured that the comparative analysis was grounded in practical reality.

The \nameref{chap:Threats} chapter detailed the range of adversarial strategies that ACNs must defend against, from passive and active attacks to broader limitations such as economic sustainability and the ongoing arms race with censorship. Understanding these threats is essential for any serious evaluation of network suitability.

The core of the thesis was the \nameref{chap:Analysis} chapter. Here, the major ACNs today: Tor, I2P, Lokinet, and Nym, were evaluated side-by-side, both through literature-based comparison and original experimental measurements. Each network was assessed in the context of the defined use case categories. The results show that each ACN has its own strengths: Tor is most effective for web browsing and infrastructure security, I2P is best for file sharing, Lokinet performs the best in low-latency use cases, and Nym offers the highest levels of anonymity. There is no configuration or network that achieves all goals simultaneously.

The chapter \nameref{chap:Future} built on the findings of the comparative analysis to suggest concrete next steps both for the field as a whole and for each network individually. Key recommendations include general guidance for all major ACNs, as well as targeted improvements for each ACN separately, based on their strengths and weaknesses that were defined in the comparative analysis.

A practical outcome of this work is the educational demonstrator described in the \nameref{chap:EducationalDemonstrator} chapter. The demonstrator puts the conclusions of the thesis into practice, enabling students to observe the strengths and weaknesses of each network in various scenarios through carefully designed laboratory exercises. This bridges the gap between theory and hands-on understanding.

To ensure continued progress in the field of anonymous communication networks and their usability, it is necessary to maintain an ongoing and critical evaluation of new ACN designs as they emerge, extending the comparative analysis to next-generation systems, and updating the set of practical use cases as technology and threats evolve. While this thesis provides a decent overview as of today, the networks will most definitely evolve and most likely new ones will appear, along with newer and improved designs, which will also need to be evaluated in terms of their usability in various use case scenarios.

In terms of the development of ACNs, research should mainly focus on following the directions defined in the \nameref{chap:Future} chapter, as well as addressing the issues defined in the \nameref{chap:Threats} chapter.

To conclude, this thesis demonstrates, both in theory and in practice, that there is no universal anonymous communication network suitable for every use case. The optimal choice always depends on a clear understanding of the technical requirements, threat models, and inherent limitations of each design. The results and tools presented here provide a foundation for both further research and practical deployment of ACNs, making it possible to match the right network to the right scenario, rather than chasing a non-existent "one-size-fits-all" solution.
