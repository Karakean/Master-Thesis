\chapter{Summary}

The main result of this work is that there is no universal anonymous communication network that is optimal for every use case scenario. The choice of the most appropriate ACN always depends on the intended application. For example, Tor typically offers the best experience for web browsing, while networks such as I2P or Nym may be more suitable for file sharing or scenarios requiring the highest levels of anonymity. The theoretical analysis and experimental results in this thesis clearly show that the design, configuration, and even the limitations of each ACN must be considered in relation to specific requirements and use cases.
The educational demonstrator developed as part of this work enables students to observe in practice the theoretical differences between the major ACN designs. It also exposes the key vulnerabilities of the most popular networks and demonstrates how alternative designs can mitigate these weaknesses. This practical approach helps explain why certain ACNs are more effective for particular tasks and less so for others.
Overall, this thesis provides a solid foundation for anyone interested in learning about or researching anonymous communication networks.

Future research in the field of anonymous communication networks can follow several promising directions. One important area is to monitor and analyse new designs as they emerge, such as Katzenpost and other next-generation Mix-nets. Another valuable direction would be to carry out more sophisticated measurements of ACN performance and anonymity guarantees under different network configurations, including varying the number of hops or relays and studying their impact on both usability and security.
