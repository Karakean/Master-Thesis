\chapter{Use cases and application areas}
\label{chap:UseCases}

This chapter presents various use cases and application areas for anonymous communication networks (ACNs). It groups these use cases based on similar requirements and establishes criteria for evaluating ACNs in each group. This classification is crucial, as it forms the foundation for later analysis and comparison of ACNs in terms of usability between different use case groups.

\section{Identifying use cases and application areas}
Identifying the practical applications of anonymous communication networks (ACNs) is essential to understanding their relevance and effectiveness in real-world scenarios. In this section, these use cases and application areas will be identified.

In restricted countries ACNs can be used to \textbf{circumvent Internet censorship}, in order to be able to browse it freely. Among the most prominent examples of censoring countries are China and its Great Firewall \cite{gfw-china}, Iran and its extensive history of Internet censorship, along with other Middle East countries. Censorship in the Arab World especially intensified during the Arab Spring where Tor played a crucial role as a tool that was used to bypass Internet restrictions \cite{arab-spring}. There are also less-known but equally severe examples of the Internet censorships, for example from Africa \cite{africa}. Freedom House is an organisation that performs periodic measurements of Internet censorship across the world \cite{freedomhouse_internet}.
One may think that in order for the user to reach the website he would simply need to use an ACN and visit it, circumventing the censorship. While it may be true in some cases, in others it is a little more complicated. There may be some built-in security features that are non-compliant with ACNs nature. For example: If someone is accessing a website in the clearnet from the ACN, he would from time to time change his circuit which was described before. This implies the fact that the exit node will change and therefore the location from which the user visits the website. While in the ACNs such a location change is normal, it may seem suspicious for a web service as it may be associated with a hacker attack.
The example of location issues described above was a real case of Facebook and an issue that legitimate users faced while reaching it via the Tor network. They could also count how many people use Facebook via Tor. In April 2016, less than two years after the first introduction of the onion service which took place in October 2014, the number of users per month passed one million \cite{facebook-tor-note}. Not only Facebook did create an onion service but they also used single hops. As it is not a secret what is the real address of Facebook's servers, they reduced the number of nodes (from the server side) necessary for the end-to-end user-server communication, reducing delays \cite{facebook-tor}. Additionally, Facebook provided an SSL certificate for their onion address. The benefits of such action were provided in the onion services section.

Censorship is not the only issue in the web browsing. Another prominent one is surveillance and ACN can provide \textbf{surveillance protection in web browsing}; therefore protect its users against monitoring their web activities. An external observer, i.e. an ISP, will only see that a specific user is using the ACN, unless the additional obfuscation is provided. In that case, the ISP can see various masking traffic, unrelated to the real actions of the user. 
While protection against censorship is challenging, protection against surveillance is even more difficult. In case of censorship the feedback loop is straightforward: certain websites are either reachable or blocked. Surveillance, on the other hand, is much harder to detect, and users are usually unaware that they are being monitored.

Getting unbiased results is crucial when it comes to research in order to ensure access to diverse, accurate and not manipulated information, free from personalisation or censorship. ACNs help in \textbf{unbiased research} by masking user identity and location, preventing search engines from tailoring results based on user’s past behaviour, geography, or profile data, therefore promoting objectivity and neutrality in search outcomes.

ACNs that support clearnet connections could, in theory, could be considered as a \textbf{VPN alternative} in terms of circumventing geo-blocking or simply virtually changing location for various reasons. However, as was described before, there is a major difference between these two: VPN providers in most cases are single-hop proxies and provide so called privacy by policy, meaning that a user has to trust some entity with his privacy and hope that they will stick to their policies, while ACNs’ privacy by design approach ensures that if any entity within the network or outside of it would like to reveal the user’s identity, it is not able to do that.
Another benefit is the fact that many ACNs are open-source and free. In the case of VPN providers they are usually paid services and if a user from a restricted country is purchasing such service from within that country it may draw the attention of the censoring government \cite{russia}. On the other hand, if the VPN service is free and at the same time it is not funded by volunteers and various organisations - it is highly likely that such a company sells the data of its users as running infrastructure costs money.

One of the lesser known advantages of hidden services is the fact that they can be used for \textbf{remote machine access}. Moreover, thanks to NAT punching the port forwarding can be avoided, which would not be possible in case of using clearnet alternatives that require opening an appropriate port. In other words, a hidden service can be set on any machine behind a firewall to reach it regardless of open ports, at the same time providing additional security benefits.

The IoT refers to the concept of connecting everyday physical objects to the internet so that they can collect, share, and exchange data. These objects can be anything from household appliances and wearable devices to industrial machinery and cars. Each device can communicate with other devices or systems over the internet or other networks. IoT security is crucial, especially when it comes to critical infrastructure. What is even more concerning is the fact that IoT manufacturers often underestimate the role of security in their devices, which results in lack of data encryption or insufficient updates. Considering the fact that these devices are often exposed to the Internet without proper authorisation or confidentiality, a conclusion can be drawn, that the problem is in fact exceptionally serious.
\textbf{Utilising ACNs with IoT} can provide enhanced security as devices are not exposed to the internet, the fact of communication between the devices or between devices and computing centers can be hidden.

A use case that is getting increasingly popular is \textbf{using ACNs to run cryptocurrency nodes}. Nodes can communicate within the network, protecting their real IP address. Thanks to such an operation, it is impossible to censor these nodes nor to shut them down as their location is not known. Moreover utilisation of hidden services can make the node setup easier and safer as no ports on a firewall need to be forwarded. Examples of cryptocurrencies that support running nodes over ACNs are Bitcoin, the most popular cryptocurrency, or Monero, widely recognised as the most private of the cryptocurrencies. Their nodes can be run within clearnet, the Tor network, the I2P network or any combination of them. In order for the nodes that run within ACNs to connect to the clearnet nodes, they need to use bridge nodes that support both ways of communication. Moreover, users that utilise wallets over ACNs can anonymously connect to ACN-specific nodes and the fact of the communication between these parties is hidden as well as their identities behind the onion addresses. Additional benefit is related to the hole punching as there is no need to open ports necessary to communicate with the nodes.

\textbf{Web-based cryptocurrency wallets} can leverage hidden services to enhance privacy and security for their users thanks to the security properties that come with them. An additional benefit from using hidden service for such use cases is that there is no risk from the exit node as the traffic never exits from the network. Malicious exit node could, for example, manipulate the cryptocurrency address in such a way that the cryptocurrency would be transferred to the attacker instead of the legit receiver. Of course such malicious activity of the exit node would be most likely recognised by the ACNs’ security mechanisms, however it is better to prevent even the tiny risk of such an event, especially considering the fact of irreversibility of the cryptocurrency transactions and potential losses.

\textbf{Email} is a fundamental communication medium which can largely benefit from the utilisation of ACNs, especially in case of activists, people under surveillance or whistleblowers.

ACNs are a perfect tool for journalists to communicate with their sources safely and anonymously. An example of a solution that utilises ACNs for \textbf{whistleblowing} is SecureDrop, which is a free and widely recognised platform for such communication. It utilises the Tor network in order to provide anonymity of the source, which is one of the most fundamental aspects of the free press, as was recognised by The European Convention of Human Rights (ECHR). Thanks to the anonymity by design it is impossible for journalists to reveal the whistleblowers identity, no matter how much pressure the government or other entities put.
Another widely recognised platform for whistleblowing is WikiLeaks. It’s a non-profit organisation and publisher of leaked documents founded in 2006. WikiLeaks releases vast numbers of documents exposing human rights violations and abuses by various entities, including governments. To protect whistleblowers’ anonymity, they use a Tor onion service as a secure submission platform. While anonymity is also preserved when accessing regular websites with Tor, onion services provide an additional layer of security desired by WikiLeaks’ creators: these sites are inaccessible without Tor, reducing the risk of exposure if submissions occur outside of the Tor network.
GlobaLeaks is a free, open-source and internationalised software that enables creation of whistleblowing initiatives. The GlobaLeaks project started in 2010 and the first version was released in 2011. It uses Tor onion services for the anonymous submissions. GlobaLeaks is a framework that was adopted by more than 10000 projects \cite{globaleaks}. The use cases of the projects implementing GlobaLeaks include independent media, activists, governments, public agencies and many more. One of the examples is WildLeaks \cite{wildleaks} - the world’s first whistleblowing initiative for environmental and animal-abuse crimes.

There are several risks that can pose a threat to the people while accessing regular websites. For example malicious local DNS servers can redirect users to other IP addresses than expected. Another possibility is from the certificate authority site where fraudulent digital certificates can be used to perform Man-in-the-middle attacks. Such events occurred in the past, Turkish CA TURKTRUST being one such example \cite{turkish-ca}. BGP hijacking is yet another issue that can be used against users accessing the websites, which is a takeover of IP addresses by corrupting BGP protocol used in the Internet. 
In order to address such issues there is a simple solution: to create a hidden service to \textbf{securely access websites}. There would not be a risk from the malicious DNS servers. Even though certificate authorities are not present, an onion service still provides end-to-end encryption and server authentication.

Anonymous and \textbf{secure one-to-one file transfer} may be beneficial in numerous situations, for example a doctor securely sending patient’s data to him or an organisation and its employees sending sensitive information between each other internally. The advantage is that there is no need of trusting any third party service provider like DropBox or WeeTransfer nor there is need to physically pass sensitive files stored on data storage devices.
An example of such a solution in Tor is OnionShare. A user that wants to send a file creates an onion URL which allows the other user to fetch the file, with a possibility to enable URL expiration so that it would be impossible to download the file again after it is downloaded once. This guarantees that the file was not downloaded by any malicious middleman. Moreover the service can additionally require a private key to access the file. In I2P it can be achieved with numerous communicators with file sharing capabilities or eepsites secured with private key that disallow unauthorised access.

Users can distribute files over ACNs in \textbf{one-to-many file publishing} manner to publish documents, media or software without the censorship. One example from history where such a use case was beneficial is the Zyprexa scandal \cite{zyprexa}. Documents were published online alleging that the pharmacy company Eli Lilly was aware of the severe side effects associated with its antipsychotic drug Zyprexa and continued to sell it regardless. Initially a pharmacy giant was able to quickly remove the documents from the websites where they were hosted, until it was moved to a hidden service. It was no longer possible for the company to remove uncomfortable documents and people had the chance to read them.

\textbf{Container registries} can have several benefits from utilising ACNs, hidden services in particular; it can prevent censorship of the container images, protects against registry spoofing attacks, increasing security of pulling images, prevents tracking of contributors in case they are not comfortable with revealing their identities and on top of that, hidden services can provide end-to-end encryption and server authentication.

Users can \textbf{communicate asynchronously} over anonymous communication networks in a privacy-preserving manner. As the communication is asynchronous it can accept longer delays but it requires the possibility of replying. There is also a possibility of \textbf{instant messaging} over ACNs, however such a use case would require low latency. Going a step further, ACNs could be used for anonymous \textbf{voice calls} or even \textbf{video calls}, hiding not only the identity of both parts of the communication, but more importantly - the fact that two parties are communicating.

Utilising ACNs with \textbf{software updates} can prevent numerous attacks as it is not possible to tell who is performing an update. On top of that utilising hidden services can provide end-to-end encryption and server authentication for the users that are performing the software updates.

Collecting usage data in privacy-preserving fashion is crucial. ACNs can be used to anonymously collect \textbf{statistics} to enirely unlink the user from the usage data he is sending, unless the data contains identifiable information that should also be removed. Utilising ACNs can allow for the large-scale analysis while preserving users' privacy.

Utilising ACNs in \textbf{gaming} can help to circumvent the censorship of the certain game, protect users from revealing their private data which leakage may lead to doxxing, or protect IP addresses of players to avoid DoS attacks. Moreover games that are not widely considered harmful, such as chess, can be forbidden in certain countries \cite{afghanistan-chess}; therefore protecting the game server in this scenario can also be justified.

Location hiding of web services can help to prevent DoS attacks, including DDoS. The attack can occur when the service content is not necessarily in accordance with the regime government and therefore is exposed to DDOS attacks from their site. ACNs can be therefore used to host \textbf{censorship and denial of service-resistant archives}. Instead of attacking a single server, an attacker would be forced to attack at least several introduction points. One of the first propositions of such systems that were censorship-proof by design were Eternity Service and Free Haven described before, however the most popular archive today is The Web Archive which decided to create a hidden service for its archive as the website was targeted with numerous attacks.

Anonymous communication networks can be used for \textbf{source code hosting}, as code repositories hosted within the ACNs can benefit from the censorship resistance, while people fetching the repositories can benefit from end-to-end encryption, server authentication, and the fact of hiding communication parties, as well as protecting against targeted attacks for users accessing certain repositories.

\textbf{Highly-available distributed data storage systems} can further enhance their censorship-resistant features by utilising ACNs. It would also protect the publishers of these files as well as the people fetching them.

While it is not a secret that companies sell the usage data of their users, especially while web browsing, which highly violates their privacy. As the users’ data is highly valuable for advertisers, the process of collecting the data is evolving. Utilising ACNs can help to \textbf{protect the privacy while web browsing}.

Anonymity of the voters is one of the most important characteristics of democratic elections. It is crucial to ensure unlinkability of a given vote to its voter and using ACNs can help with it. ACNs could have been utilised for \textbf{e-voting systems} once the traditional voting will be replaced or complemented by online solutions.


\section{Categorising use cases}
The use cases identified in the previous section can be grouped into several groups based on the requirements they have in terms of the ACNs. Anonymity is a basic requirement for these use cases, but the strength of provided anonymity can be evaluated within criteria. Each group will be declared in terms of what kind of anonymity is needed for this group: sender, receiver (server), or both. Distinguished will be described in the following subsections.

\subsection{Low-latency intra-network communication}
Use cases and application areas for which the lowest possible latency is crucial. It includes instant messaging, gaming, voice, and video calls.
As these use cases are focused on two-way communication along with the fact that giving up one-end anonymity may impose additional threats from the traffic analysis perspective with low-latency communication, the comparison will be based only on solutions providing two-way anonymity, meaning that the communication has to be performed within the network. While both sides of communication are often known to each other, the importance of such a property lies in the lack of exposing metadata - who is speaking to whom.
Criteria defined for this group are visualised in the Table~\ref{tab:low_latency_criteria}.

\begin{table}[!ht]
\centering
\caption{Criteria and weights for low-latency intra-network communication use cases}
\begin{tabular}{|l|c|}
\hline
\textbf{Criterion} & \textbf{Weight} \\
\hline
Low latency & 0.25 \\
Low jitter & 0.25 \\
Throughput & 0.20 \\
Both-end anonymity strength & 0.15 \\
UDP support with low packet loss & 0.15 \\
\hline
\end{tabular}
\label{tab:low_latency_criteria}
\end{table}

If one-way transmission delay is too high, the applications mentioned above cannot be achieved. Therefore, weight of 0.25 was assigned to the low latency criterion. According to ITU-T recommendations \cite{ITU-G.114} maximum one-way latency for international calls is 400 ms, ideally 150 ms or less.

The great variation between subsequent packet arrivals can significantly reduce perceived connection quality \cite{voip-tor}, and similarly to latency,  making use cases of this group not feasible to deploy. This is the reason why jitter was also assigned with a weight of 0.25. The lower the value for jitter, the better the connection quality is; the value should be less than 30 ms.

Sufficient throughput is needed for transmitting video and audio frames, especially those of high quality. One of the least demanding voice codecs is G.729 which requires only 8 kbit/s of throughput \cite{ITU-G.729}. When it comes to video calls, Skype required \cite{microsoft-skype-bandwidth} a minimum of 128 Kbit/s, and the recommended throughput for high quality video calls was 500 Kbit/s. The higher the throughput, the higher the quality of the video and audio frame can be. As it is also an important property for the use case scenarios, it was assigned a strong weight of 0.2.

While the stronger anonymity, the better it is for ACN users, according to the literature it is not possible to provide low latency, high bandwidth, and strong anonymity at the same time \cite{anonymity-trillema}. Choosing stronger anonymity over the latency or bandwidth would make use cases from this use case group irrelevant, therefore anonymity was assigned a value of 0.15, which is a smaller weight than the previous criteria, although anonymity is obviously still relevant and was not omitted.

In general, low-latency applications and protocols rely on the UDP protocol, as the overhead of TCP is considered too impractical. Examples of such protocols include QUIC and RTP. Even if TCP could have provided decent performance, running a service via ACN should not require major architectural changes for popular solutions used today. While TCP handles packet loss underneath, UDP does not. However, packet loss is more acceptable than high latency or jitter \cite{voip-tor}, it is still an important property that should be considered within this category. It was assigned a weight according to its importance in this category of use cases.  Taking into account all of these aspects, a weight of 0.15 was assigned to UDP support with low packet loss.

\subsection{Highest anonymity and latency-tolerant}
Use cases and application areas for which latency is not crucial and can be traded for potential anonymity benefits. Those include emails, whistleblowing, asynchronous messaging, statistics, and e-voting systems.
As the use cases are mainly focused on the sender anonymity, only this type of anonymity will be considered within this category.
Criteria defined for this group are visualised in the Table~\ref{tab:high_anonymity_criteria}.

\begin{table}[!ht]
\centering
\caption{Criteria and weights for highest anonymity and latency-tolerant use cases}
\begin{tabular}{|l|c|}
\hline
\textbf{Criterion} & \textbf{Weight} \\
\hline
Sender anonymity strength & 0.5 \\
Reliable delivery & 0.25 \\
Message persistence & 0.25 \\
\hline
\end{tabular}
\label{tab:high_anonymity_criteria}
\end{table}


As low latency, high bandwidth and strong anonymity cannot be combined at once \cite{anonymity-trillema}, and considering the fact that currently there are no low-latency networks providing strong anonymity (they could have been potentially based on DC-nets) in this use case group the anonymity can be chosen with the cost of larger delays, providing strongest metadata protection, and therefore can be considered as the main focus within the category, receiving the largest criteria weight of 0.5. It also allows for the network to be more resilient against traffic analysis attacks, which will be described in the next chapter, that can be performed by a powerful global passive adversary - potentially the one that a whistleblower wants to reveal information about.

It is crucial for the user to determine whether his message has been delivered due to the importance of the information that is often associated with this use case group. The lack of a mechanism that addresses this issue, such as an acknowledgment, is a significant drawback that was present in early Mix-net ACNs \cite{mix-net-reliability}; therefore, the reliable delivery criterion was assigned a weight of 0.25.

Asynchronous communication can often involve parties that are temporarily offline. It is therefore a decent feature for the network to have the built-in possibility of temporarily storing messages; otherwise, the message has to be sent to an always-online entity within the ACN, possibly an offline entity outside of the ACN, or a service provider must create such a mechanism on top of the network, if there is such a possibility. The lack of these possibilities also imposes security issues as described in the literature \cite{two-cents-post-office}. Due to these factors, a weight of 0.25 was assigned to the message persistence.

\subsection{Web browsing-based}
In this category, there are use cases that are strictly related to web browsing: Internet censorship circumvention, surveillance protection in web browsing, unbiased research, VPN service alternative, web-based cryptocurrency wallets, secure accessing websites, and protecting privacy in web browsing.
Within this category, in the comparison the sender's anonymity will be considered mainly, as it is a category highly focused on users.
Criteria defined for this group are visualised in the Table~\ref{tab:web_browsing_criteria}.

\begin{table}[!ht]
\centering
\caption{Criteria and weights for web browsing-based use cases}
\begin{tabular}{|l|c|}
\hline
\textbf{Criterion} & \textbf{Weight} \\
\hline
Clearnet support & 0.25 \\
Censorship resistance & 0.25 \\
Latency & 0.2 \\
Sender anonymity strength & 0.2 \\
Web browsing optimisation & 0.1 \\
\hline
\end{tabular}
\label{tab:web_browsing_criteria}
\end{table}

Currently there are over 1.5 billion websites on the world wide web and less than 200 million of them are active \cite{internetlivestats}. In Tor, the most popular anonymous communication network, there are around 800 thousand unique onion addresses \cite{tor-metrics, tor-metrics2} and not every onion address is associated with a website, as there are many other use cases for them. Therefore, the vast majority of websites are not reachable as hidden services, requiring for this specific use case group to allow users for exit traffic. That is why the clearnet support criterion was assigned the most significant weight of 0.25.
    
As many countries impose censorship on ACNs, Tor in particular as the most popular network \cite{gfw-china, africa}, it is crucial that the network address censoring so that users in these countries can use ACNs freely for web browsing-based use cases. The censorship resistance was for that reason considered as one the second most important features in this use case scenarios, receiving a notable weight of 0.25.
    
Since Web browsing is an interactive activity, it requires little latency. Although the lower the value, the better it is for the user, the anonymity must be preserved. It implies a trade-off between those properties as described in the literature \cite{tor-design}. Due to the significance of this property in terms of comfortable web browsing, the low latency criterion was assigned a weight of 0.2.
    
As this use case group does not aim for the strongest anonymity that cannot be achieved without additional latency, a trade-off must be accepted. For that reason, sender anonymity strength is considered as a criterion with significant weight of 0.2, but not more significant than latency.
    
ACN optimisation in terms of web browsing such as ease of setup, browser fingerprinting resistance, HTTPS support, or others. While it may bring additional benefits for utilisation of a given ACN in web browsing, it is not a critical functionality, and therefore, was assigned the lowest weight of 0.1.


\subsection{File sharing-based}
This category involves use cases and application areas related to file sharing, including: one-to-many file publishing, one-to-one file transfer, anonymous container registries, anonymous software updates, source code hosting, and distributed data storage systems.
Within this use case group, both-end anonymity, including mutual anonymity (parties of the communication are not known to each other), will be considered as in these use cases, anonymity is needed for both sides of communication.
Although in theory this category could also compare anonymous publication systems, they can fulfil only the minority of the mentioned use cases; therefore, they would not be taken under consideration.
Criteria defined for this group are visualised in the Table~\ref{tab:file_sharing_criteria}.

\begin{table}[!ht]
\centering
\caption{Criteria and weights for file sharing-based use cases}
\begin{tabular}{|l|c|}
\hline
\textbf{Criterion} & \textbf{Weight} \\
\hline
Download speed & 0.3 \\
Both-end anonymity strength & 0.3 \\
Volume correlation resistance & 0.3 \\
File sharing protocols support & 0.1 \\
\hline
\end{tabular}
\label{tab:file_sharing_criteria}
\end{table}

Download speed is critical for transferring large files. Although the higher the download speed, the better, the high speeds that are known from non-ACN-based file sharing are not achievable in ACNs. Due to its intuitively large significance in this group, it was assigned a weight of 0.3.

Due to the importance of anonymity not only for the receiver/downloader but also sender/publisher as described in the literature \cite{rewebber, freehaven}, a both-end anonymity strength was assigned a weight of 0.3.

While volume correlation resistance can be considered as a part of anonymity strength, a separate criterion was distinguished due to its importance in this group of usage scenarios with appropriate weight of 0.3. Large volume transfers create an identifiable pattern that can be easily distinguished by an attacker if there are no additional protections against it \cite{tor-design}.

There are numerous file sharing protocols that optimise the download process, for example, BitTorrent. The ability to deploy them in an ACN without major modifications is considered a desirable property, although not a critical one, and therefore assigned a weight of 0.1.

\subsection{Infrastructure security and resilience-based}
In this category, the highest importance is associated with the security and resilience of the underlying infrastructure. It includes remote access to hosts, Internet of Things (IoT), running cryptocurrency nodes, as well as censorship and denial-of-service resistant archives. As these use cases are primarily focused on server-side anonymity, only ACNs providing such functionality will be considered. Criteria defined for this group are visualised in the Table~\ref{tab:infrastructure_security_and_resilience_criteria}.

\begin{table}[!ht]
\centering
\caption{Criteria and weights for infrastructure security and resilience-based use cases}
\begin{tabular}{|l|c|}
\hline
\textbf{Criterion} & \textbf{Weight} \\
\hline
Resilience to active attacks & 0.3 \\
Authorisation mechanisms & 0.3 \\
Server anonymity strength & 0.2 \\
Academic research and maturity & 0.2 \\
\hline
\end{tabular}
\label{tab:infrastructure_security_and_resilience_criteria}
\end{table}

Long-running services within this category must have high resistance to active attacks provided by the ACN, especially disruptive DoS ones that may cause decreased availability of the service, as was the case with an Internet Archive \cite{ddos-web-archive}. The ACN must be highly resistant to these types of attacks; therefore, it was assigned a notable weight of 0.3 was assigned.

One of the benefits of utilising ACNs in this use case group is the possibility of omitting exposing an infrastructure to the clearnet. As hidden services across ACNs are stored in DHT structure and can possibly be enumerated \cite{Owen2016}, additional authorisation mechanisms to prevent direct access to hidden services are desired - otherwise, these mechanisms need to be implemented on the application side. Due to its importance, it was assigned a weight of 0.3.

In this use case group, providing server anonymity is crucial. The hiding location may positively correlate with the resilience to active attacks mentioned earlier. Anonymity can also serve to protect critical infrastructure from discovery \cite{torusers}. Taking into account these aspects, a weight of 0.2 was assigned to the anonymity of the server.

Academic research directly influences the resilience of the network as it investigates various aspects, potential threats, and attacks on a given ACN and gives more confidence in its safety. Maturity is directly correlated with smaller chances of existing bugs, as many of them were discovered through the years of existence of a given ACN. The reasons for this are similar to the Kerckhoffs principle \cite{kerckhoffs}. The more a system is studied and tested openly, the less it relies on secrecy to stay secure. Instead, security comes from technical aspects, examined by many experts and used over time, which helps to find and fix problems. So, academic research and maturity make an ACN safer because they lead to more thorough testing and fewer hidden bugs. The academic research and maturity criterion is therefore assigned with a significant weight of 0.2.

\section{Summary}
In summary, this chapter presented a detailed list of use cases and application areas for anonymous communication networks, grouped them according to shared requirements, and defined criteria for their evaluation. The role of this chapter in the thesis is to connect technical aspects with practical reality, ensuring that the later analysis of ACNs is focused on real user needs. As such, this chapter provides the necessary basis for comparative analysis and clarifies what exactly needs to be compared and why.
